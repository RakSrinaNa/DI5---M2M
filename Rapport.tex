\documentclass{report}
\usepackage{MCC}

\def\footauthor{Thomas COUCHOUD \& Victor COLEAU}
\title{TP M2M}
\author{Thomas COUCHOUD\\\texttt{thomas.couchoud@etu.univ-tours.fr}\\Victor COLEAU\\\texttt{victor.coleau@etu.univ-tours.fr}}

\begin{document}
	\mccTitle
	
	\chapter{Prise en main}
		\section{Faire clignoter une LED}
			Controller la LED ne pose pas de grand problème.
			Il faut juste bien penser à initialiser les différents éléments (Serial, port de la LED).
			
			Le code utilisé est disponible en \autoref{code:led}.
			
		\section{LED RGB}
			Afin de pouvoir utiliser la LED RGB, nous utilisons la librairie ChainableLED.
			Cette dernière nous permet de créer des objets ChainableLED.
			Pour l'initialiser nous donnons la broche de l'horloge, la broche des données puis enfin le nombre de LEDs dans la chaine.
			Par exemple si nous branchons la LED en D3, nous initialisons avec led(3,4,1).
			
			Une fois l'objet créé, il ne faut pas oublier de l'initialiser dans la méthode init grâce à led.init().
			
			Dans le code nous pouvons ensuite utiliser les méthodes fournies:
			\begin{easylist}[itemize]
				@ void setColorRGB(led, red, green, blue);
				@ void setColorHSB(led, hue, saturation, brightness);
			\end{easylist}
			
  			Le paramètre led correspond à l'indice de la LED dans la chaine.
  			
  			Concernant la boucle loop, cette dernière effectue les opérations suivantes:
  			\begin{easylist}[itemize]
  				@ Récupère la valeur du potentiomètre (analogRead)
  				@ Si cette valeur est plus grande qu'un seuil, on allume une des LED, sinon on l'éteint
  				@ On map la valeur du potentiomètre entre 0 et 1 puis allumons la LED RGB avec comme intensité la valeur mappé
  				@ On attend 20ms
  			\end{easylist}
  			
  			Le code est disponible en \autoref{code:rgb}.
  			
  		\section{Température}
  			De la même manière pour la température, nous utilisons la librairie DHT.
  			Nous pouvons ensuite créer un objet DHT avec comme paramètres le port sur lequel est branché le capteur ainsi que le type du capteur utilisé.
  			
  			Il fait ensuite initialiser notre objet dans la fonction init grace à dht.begin().
  			
  			Enfin nous pouvons obtenir la température et l'humidité grâce aux fonction dht.readHumidity() et dht.readTemperature().
  			
  			Notre code fait exactement les mêmes étapes qu'avec le potentiomètre mais map la température entre 20 et 50 degrés vert une valeur entre 0 et 255 pour controller la LED en RGB.
  			Voir \autoref{code:temp}.
  			
  		\section{LCD}
  			Encore une fois pour utiliser l'écran LCD nous utilisons une librairie: rgb\_lcd.
  			L'objet à créer est un bjet rgb\_lcd.
  			Nous l'initialisons dans le setup grâce à lcd.begin(nombre de colonnes, nombre de lignes).
  			Ensuite nous définissons une couleur par défaut.
  			
  			Dans la fonction loop, nous récupérons la température et humidité puis l'affichons sur l'écran.
  			Pour cela:
  			\begin{easylist}[itemize]
  				@ On place le curseur en 0,0 grâce à set cursor
  				@ On écrit "T :"
  				@ On place le curseur en 4,0
  				@ On écris la température
  				@ On place le curseur en 0,1
  				@ On écrit "H :"
  				@ On place le curseur en 4,1
  				@ On écris l'humidité
  				@ On place le curseur en 15,1
  				@ On écris "\%"
  			\end{easylist}
  			
  			De plus nous changeons la couleur de l'écran en fonction de la température et humidité grâce à lcd.setColor(rouge, vert, bleu).
  			
  			Code disponible en \autoref{code:lcd}.
  	
  	\chapter{TP1}
  		\section{Adresse I2C}
  			Afin de récupérer l'adresse I2C du capteur, nous utilisons le I2CScanner proposé \href{https://playground.arduino.cc/Main/I2cScanner}{ici}.
  			Grâce à ce code, nous avons pu identifier que le baromètre à pour adresse 0x76.
  			
  		\section{Registre}
  			\begin{easylist}[itemize]
  				@ 0x0D: ID, contient l'ID du périphérique.
  				@ 0xE0: Reset, si on écrit 0xB6, le périphérique est réinitialisé.
  				@ 0xF2: ctrl\_hum, permet de définir les options de mesure de l'humidité. Le registre devient effectif après une écriture dans ctrl\_meas.
  				@ 0xF3: status, contient 2 bits indiquant le status du périphérique.
  				@@ Bit 3: mis à 1 quand une conversion est en cours, et 0 quand les résultats ont étés transférés.
  				@@ Bit 0: mis à 1 quand des données NVM sont copiés dans l'image du registre et 0 quand le transfert est fini.
  				@ 0xF4 ctrl\_meas: enregistre les données de capture de pression et température.
  				@ 0xF5 config: définis des options supplémentaires.
  				@ 0xF7…0xF9 press (\_msb, \_lsb, \_xlsb): contient les valeurs non modifiées des mesures de pression.
  				@ 0xFA…0xFC temp (\_msb, \_lsb, \_xlsb): pareil mais pour la température.
  				@ 0xFD…0xFE hum (\_msb, \_lsb): pareil mais pour l'humidité.
  				\end{easylist}
  			
  		\section{Librairie}
  			La librairie comporte un problème, le port I2C utilisé est 0x77 et non pas 0x76.
  			Nous avons donc du changer ce paramètre.
  			
  			On commence par déclarer un objet Adafruit\_BME280 puis on l'initialise avec bme.begin().
  			
  			Afin d'obtenir les mesures physiques, nous avons accès à:
  			\begin{easylist}[itemize]
  				@ bme.readTemperature()
  				@ bme.readHumidity()
  				@ bme.readPressure()
  				@ bme.readAltitude(SEALEVELPRESSURE\_HPA) où le paramètre correspond à la pression à l'altitude 0.
  			\end{easylist}
  			
  			Puis on affiche les données sur l'écran LCD.
  			Le code est disponible en \autoref{code:bme}.

			
\appendix
	\chapter{Prise en main}
		\section{Faire clignoter une LED\label{code:led}}
			\lstinputlisting[caption=led.ino, language=JAVA]{"TP M2M/led/led.ino"}
			
		\section{LED RGB\label{code:rgb}}
			\lstinputlisting[caption=led\_rgb\_pot.ino, language=JAVA]{"TP M2M/led_rgb_pot/led_rgb_pot.ino"}
			
		\section{Température\label{code:temp}}
			\lstinputlisting[caption=led\_rgb\_temp.ino, language=JAVA]{"TP M2M/led_rgb_temp/led_rgb_temp.ino"}
			
		\section{LCD\label{code:lcd}}
			\lstinputlisting[caption=lcd.ino, language=JAVA]{"TP M2M/lcd/lcd.ino"}
			
		\section{Baromètre\label{code:bme}}
			\lstinputlisting[caption=bme.ino, language=JAVA]{"TP1/bme/bme.ino"}

\end{document}